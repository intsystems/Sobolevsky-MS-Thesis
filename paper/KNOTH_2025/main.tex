\documentclass[a4paper,10pt]{article}
\usepackage[
   top=23mm,
   bottom=23mm,
   left=32mm,
   right=32mm]{geometry}
\usepackage[]{cite}
\usepackage{cmap}
\usepackage[T2A]{fontenc}
\usepackage[utf8]{inputenc}
\usepackage[english, russian]{babel}
\usepackage{amsmath, amsfonts,amssymb}
\usepackage{graphicx, epsfig}
% \usepackage{subfig}
\usepackage{subcaption}
\usepackage{color}
\usepackage{hyperref}


% \newcommand\argmin{\mathop{\arg\min}}
% \newcommand{\T}{^{\text{\tiny\sffamily\upshape\mdseries T}}}
% \newcommand{\hchi}{\hat{\boldsymbol{\chi}}}
% \newcommand{\hphi}{\hat{\boldsymbol{\varphi}}}
% \newcommand{\bchi}{\boldsymbol{\chi}}
% \newcommand{\A}{\mathbf{A}}
% \newcommand{\bb}{\mathbf{b}}
% \newcommand{\B}{\mathcal{B}}
% \newcommand{\W}{\mathbf{W}}
\newcommand{\E}{\mathbb{E}}
\renewcommand{\P}{\mathbb{P}}
\newcommand{\x}{\mathbf{x}}
\newcommand{\y}{\mathbf{y}}
\newcommand{\Y}{\mathbf{Y}}
\newcommand{\X}{\mathbf{X}}
\newcommand{\D}{\mathcal{D}}
\newcommand{\T}{\mathcal{T}}
\newcommand{\s}{\mathbf{s}}
% \newcommand{\Z}{\mathbf{Z}}
% \newcommand{\hx}{\hat{x}}
% \newcommand{\hX}{\hat{\X}}
% \newcommand{\hy}{\hat{y}}
\newcommand{\M}{\mathcal{M}}
\newcommand{\I}{\mathcal{I}}
\newcommand{\Q}{\mathcal{Q}}
\renewcommand{\S}{\mathcal{S}}
% \newcommand{\N}{\mathcal{N}}
\newcommand{\R}{\mathbb{R}}
% \newcommand{\p}{p(\cdot)}
% \newcommand{\cc}{\mathbf{c}}
% \newcommand{\m}{\mathbf{m}}
% \newcommand{\bt}{\mathbf{t}}
% \newcommand{\e}{\mathbf{e}}
% \newcommand{\h}{\mathbf{h}}
% \newcommand{\q}{q(\cdot)}
% \newcommand{\uu}{\mathbf{u}}
% \newcommand{\vv}{\mathbf{v}}
% \newcommand{\dd}{\partial}

\renewcommand{\baselinestretch}{1}


\newtheorem{Th}{Теорема}
\newtheorem{Def}{Определение}
\newenvironment{Proof} % имя окружения
    {\par\noindent{\bf Доказательство.}} % команды для \begin
    {\hfill$\scriptstyle\blacksquare$} % команды для \end
\newtheorem{Assumption}{Предположение}
\newtheorem{Corollary}{Следствие}

\usepackage{fontspec}
% \usepackage{polyglossia}
% \setmainlanguage{russian}

\setmainfont{Times New Roman}
% \newfontfamily\cyrillicfont{Times New Roman}[Script=Cyrillic]

\usepackage{sectsty}
\sectionfont{\fontsize{14}{14}\bfseries\selectfont}
\subsectionfont{\fontsize{12}{12}\bfseries\selectfont}

\parindent=24pt % абзацный отступ
\usepackage{setspace}
\singlespacing % интервал между абзацами
\usepackage{indentfirst} % отступ для первых абзацев разделов
\pagestyle{empty} % страницы без нумерации
% \tolerance=2000 % терпимость к "жидким" строкам
% \flushbottom % выравнивание высоты страниц

%\graphicspath{ {fig/} }



\author{ Соболевский Ф.\,А., Воронцов К.\,В.}
\date{\today}

\begin{document}



\begin{center}
{\fontsize{11}{11}\selectfont \textbf{\underline{Знания-Онтологии-Теории (ЗОНТ-2025)}}}
\vspace{1cm}

{\fontsize{20}{20}\bfseries\selectfont \textbf{Расстояние редактирования текстового дерева: сравнение текстовых иерархий с использованием языковых моделей.}}

\vspace{1cm}
{\fontsize{12}{12}\selectfont Соболевский Федор Александрович$^1$, Воронцов Константин Вячеславович$^{1,2}$}
\vspace{0.75cm}

\textit{$^1$Московский физико-технический институт, Керченская улица, 1А, корп. 1, г. Москва, 117303, Россия.}

\textit{$^2$Московский государственный университет имени М. В. Ломоносова, Ленинские горы, д. 1, г. Москва, 119991, Россия.}

\vspace{0.75cm}
sobolevskii.fa@phystech.edu, k.v.vorontsov@phystech.edu
\vspace{0.75cm}


\end{center}

\noindent\textbf{Аннотация.} \textit{В задачах автоматической иерархической суммаризации текстов возникает необходимость в сравнении текстовых иерархий для оценки качества. Для этого, однако, на данный момент не существует общепринятых метрик, а используемые в предыдущих работах методы оценки качества либо основаны на ручной проверке, либо слабо учитывают структуру и семантику текстовых иерархий. В связи с этим мы предлагаем использовать для сравнения текстовых деревьев расстояние редактирования текстового дерева (text tree edit distance, TTED). Данная метрика основана на расстоянии редактирования дерева, в которой стоимость операций редактирования определяется через семантическое расстояние между текстами в вершинах, аппроксимируемое с помощью выбранной большой языковой модели. TTED призвана отражать прежде всего значимые различия между текстовыми иерархиями, моделируя одновременно как структурные, так и семантические различия между ними. С помощью грамотного подбора модели-энкодера в TTED можно добиться высокого качества сравнения текстовых деревьев между собой. Практическая реализация предложенного нами алгоритма доступна по ссылке на Github:}
https://github.com/intsystems/text-tree-distance.

\vspace{0.75cm}
\noindent\textbf{Ключевые слова: текстовые деревья, расстояние редактирования, большие языковые модели, алгоритм Чжана-Шаши.}

\section{Введение}
Текстовые деревья~--- деревья, в вершинах которых находятся тексты~--- как структура данных возникают в ряде различных задач: иерархической суммаризации, генерации интеллект карт и др. При разработке методов автоматической генерации подобных структур возникает потребность в оценке качества сгенерированных иерархий. Стандартным подходом к автоматическому оцениванию качества в таких задачах является сравнение с золотым стандартом, созданным вручную экспертом, однако в случае текстовых иерархий для этих целей на данный момент не существует общепринятых метрик. В работах \cite{wei2019revealing, hu2021efficient, zhang2024coreference} для оценки сходства текстовых интеллект-карт применяется функция сходства, основанная на сопоставлении ребер деревьев и сравнении текстов в вершинах при помощи семейства метрик ROUGE \cite{lin2004rouge}. Такой подход, однако, слабо учитывает как конкретику структурных различий между деревьями, так и семантику текстов в них \cite{fabbri2021summeval}. Это и отсутствие других воспроизводимых метрик для оценки качества генерации текстовых иерархий обуславливает необходимость создания нового метода сравнения текстовых деревьев между собой.

На сегодняшний день существует немало способов сравнивать между собой деревья и тексты по отдельности. Для сравнения структур деревьев применяются такие метрики, как коэффициент Жаккара \cite{jaccard1901distribution}, расстояние Робинсона-Фоулдса \cite{robinson1981comparison} и расстояние редактирования дерева \cite{zhang1989simple}. Для моделирования семантики текстов на сегодняшний день успешно применяются большие языковые модели~--- в частности, энкодеры на основе архитектуры BERT \cite{vrbanec2023comparison}. Наша работа призвана объединить методы сравнения деревьев и текстов для создания метрики, позволяющей сравнивать текстовые деревья как объекты, объединяющие семантику текстов и структуру деревьев.

\section{Теоретические проблемы}

Рассмотрим одну из задач, в которой возникает потребность в сравнении между собой текстовых деревьев~--- иерархическую суммаризацию. Пусть задано множество $\mathcal{S}$ текстов над некоторым словарем. Определим текстовое дерево $T = (V, E)$, $E\subset V^2$, для каждой вершины $v\in V$ которого задан текст $s(v)\in \mathcal{S}$. Обозначим множество рассматриваемых текстовых деревьев как $\mathcal{T}$. Задача иерархической суммаризации~--- построение отображения $f: D\mapsto T$, строящего иерархическую сводку $T\in\mathcal{T}$ по документу $D$, минимально отличающуюся от эталонной сводки $T^*$, построенной экспертом по этой же сводке. Оптимизационная задача в этом случае записывается как
$$
\rho(f(D), T^*)\longrightarrow\min_f.
$$
Именно здесь и возникает проблема задания адекватной метрики $\rho: \mathcal{T}^2\rightarrow\R^+$ на множестве текстовых деревьев $\T$, отражающей, прежде всего, значимые различия текстовых деревьев. Значимыми мы будем считать различия деревьев по их структуре и по семантике (смысловому содержанию) текстов в их вершинах. В качестве незначительного различия можно выделить, например, перефразирование текстов в вершинах. Нашей целью будет предложить такую метрику, для которой расстояния между деревьями, отличающимися друг от друга только перефразированием, будут в среднем как можно меньше, чем расстояния между деревьями, отличающимися по структуре и/или семантике.


\section{Предлагаемый метод.}

Предлагается следующая метрика~--- \textit{расстояние редактирования текстового дерева}, или \textit{TTED} (text tree edit distance). TTED между двумя текстовыми деревьями определяется как расстояние редактирования дерева~--- наименьшая суммарная стоимость операций редактирования, позволяющих получить из одного дерева другое~--- со стоимостью операций редактирования, определяемой как семантическое расстояние между текстами в вершинах в случае операции замены вершины и как расстояние от текста в вершине до пустой строки в случае добавления и удаления вершины. Расстояние редактирования с заданными стоимостями операций редактирования можно эффективно вычислять для упорядоченных и неупорядоченных деревьев с помощью алгоритма Чжана-Шаши \cite{zhang1989simple, zhang1992editing}.

Для измерения семантического расстояния между текстами применим большую языковую модель-энкодер (кодировщик) $\text{LM}: S \rightarrow \R^n$. С помощью этой модели можно сопоставить текстами некоторые конечномерные векторы (эмбеддинги), расстояние между которыми уже можно измерить с помощью стандартных метрик в $\R^n$. Тогда мы можем определить для $s, s'\in\mathcal{S}$ семантическое расстояние как $r(s, s') = \rho_n(\text{LM}(s), \text{LM}(s'))$, где $\rho_n$~--- метрика в $\R^n$. Итого, стоимости операций редактирования в TTED задаются следующим образом:
\begin{enumerate}
    \item Стоимость замены вершины $v$ на вершину $v'$~--- $\rho_n(\text{LM}(s(v)), \text{LM}(s(v')))$;
    \item Стоимость удаления/добавления вершины $v$~--- $\rho_n(\text{LM}(s(v)), \text{LM}(\lambda))$, где $\lambda$~--- пустая строка.
\end{enumerate}

TTED позволяет, таким образом, сравнивать текстовые деревья, одновременно учитывая их структуру с помощью расстояния редактирования, являющегося основой TTED, и семантику с использованием больших языковых моделей для ее моделирования.

\section{Вычислительные эксперименты}
Для тестирования TTED с разными моделями-энкодерами в сравнении с базовым методом, использованным в работах \cite{wei2019revealing, hu2021efficient, zhang2024coreference}, были измерены расстояния/значения сходства на синтетической выборке, состоящей из деревьев и их модификаций. Средние расстояния между деревьями и их модификациями по перефразированию, структуре и семантике обозначим как $\overline{\rho}_1$, $\overline{\rho}_2$ и $\overline{\rho}_3$ соответственно, средние значения функции сходства~--- $\overline{\text{Sim}}_1$, $\overline{\text{Sim}}_2$ и $\overline{\text{Sim}}_3$ соответственно. Результаты тестирования TTED представлены в таблице~\ref{tab:tted_results}. Для базового метода были получены следующие результаты:
$$
\overline{\text{Sim}}_1 = 3,92\pm0,29,\quad \overline{\text{Sim}}_2 = 6,92\pm0,67,\quad \overline{\text{Sim}}_3 = 2,64\pm0,46.
$$
\begin{table}[h]
    \centering
    \caption{Средние оценки расстояния с помощью TTED с разными моделями-энкодерами}
    \begin{tabular}{c|c|c|c}
        \textbf{Модель-энкодер в TTED} & \textbf{$\overline{\rho}_1$} & \textbf{$\overline{\rho}_2$} & \textbf{$\overline{\rho}_3$} \\ \hline
        DistilRoBERTa & 3,33 & 7,76 & 7,38 \\ \hline
        SPECTER & 1,39 & 3,70 & 4,74 \\ \hline
        MPNet & 2,30 & 7,19 & 8,06 \\ \hline
        Дообученная MPNet & 1,82 & 7,71 & 7,56
    \end{tabular}
    \label{tab:tted_results}
\end{table}

По результатам тестирования можно видеть, что структурные различия отражаются базовым методом заметно слабее, чем отличия по перефразированию и семантике. Семантика и перефразирование же отражаются примерно одинаково, что неудивительно, учитывая, что сравнение текстов в вершинах в базовом методе ведётся на уровне лексических единиц с помощью ROUGE. Для сравнения, различия деревьев по структуре и семантике отражаются с помощью TTED заметно сильнее, чем отличия по перефразированию. Более того, для структурных и семантических различий значения расстояний в среднем близки, что позволяет утверждать, что TTED не только более информативно, чем базовый метод, но и сбалансировано отражает различные аспекты различия текстовых деревьев. Все это обосновывает применение метрики TTED для оценки качества в задачах автоматической генерации текстовых иерархий в дальнейшем.

\addcontentsline{toc}{section}{\protect\numberline{}Литература}
\bibliographystyle{ugost2008}
\bibliography{references}

\end{document}